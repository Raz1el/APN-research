\documentclass[11pt]{beamer}
\usetheme{Copenhagen}
\usepackage[utf8]{inputenc}
\usepackage[T2A]{fontenc}
\usepackage[russian]{babel}
\usepackage{amsmath}
\usepackage{amsfonts}
\usepackage{amssymb}
%\author{}
%\title{}
%\setbeamercovered{transparent} 
%\setbeamertemplate{navigation symbols}{} 
%\logo{} 
%\institute{} 
%\date{} 
%\subject{} 
\begin{document}

%\begin{frame}
%\titlepage
%\end{frame}

%\begin{frame}
%\tableofcontents
%\end{frame}

%------------------------------------------------------------

\begin{frame}{ВВЕДЕНИЕЕЕЕЕ}
ВВЕДЕНИЕ
\end{frame}
\begin{frame}{АЛГ}
АЛГОРИТМ ПЕРЕСЧЕТА
\end{frame}

\begin{frame}{Подходы к построению подстановок с низкой дифф. равномерностью}

Наиболее удачные подходы:

\begin{itemize}
\item Полный перебор транспозиций.
\item Перебор образующих транспозиций.
\item Перебор транспозиций, образованных элементами пересечения.
\end{itemize}

\end{frame}

%------------------------------------------------------------




\begin{frame}{``Комбинированный'' подход}

После многочисленных экспериментов оказалось, что лучше всего справляется с задачей уменьшения дифференциальной равномерности ``комбинированный'' подход, который состоит из поочерёдного применения подходов с перебором образующих транспозиций и с перебором транспозиций, образованных элементами пересечения. Такой ``комбинированный'' подход позволяет стабильно снижать дифференциальную равномерность S-блока размерности 8 до значения 6.

\end{frame}

%------------------------------------------------------------

\begin{frame}{Направления дальнейшего исследования}

\begin{itemize}
\item Придумать и опробовать новые эвристики.
\item Переписать код на более ``быстрый'' язык.
\item Использовать более мощный компьютер для перебора.
\end{itemize}

\end{frame}

\begin{frame}{ЭКСП}
ВЫЧИСЛИТЕЛЬНЫЕ ЭКСПЕРИМЕНТЫ
\end{frame}
%------------------------------------------------------------

\end{document}