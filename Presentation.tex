\documentclass{beamer}
\usepackage[T2A]{fontenc}
\usepackage[utf8]{inputenc}
%\usepackage[cp1251]{inputenc}
\usepackage[english,russian]{babel}
\usepackage{amssymb,amsfonts,amsmath,mathtext}
\usepackage{graphicx} % Подключаем пакет для вставки картинок (с драйвером dvips).
%\usepackage{color}
\usepackage{ragged2e}
\usepackage{pdfpages}
\usepackage{bm}
\usepackage{mathrsfs}
\usepackage{multirow}
\usepackage{array}
\usepackage[rightcaption]{sidecap}
\usepackage{braket}
\usepackage{mathtools}
\usepackage{tikz}
\usepackage{array, makecell}

\usepackage{array}   % Для улучшенного форматирования таблиц
\usepackage{amsmath}  % Для математических символов
\usepackage{booktabs} % Для профессиональных таблиц
\usepackage{stackengine} 
\usepackage{xcolor}

\DeclarePairedDelimiter\ceil{\lceil}{\rceil}
\DeclarePairedDelimiter\floor{\lfloor}{\rfloor}
\beamertemplatenavigationsymbolsempty
\usepackage{enumitem}
\usepackage[makeroom]{cancel}
%\usepackage[dvipsnames]{xcolor}
\usepackage{algorithm}
\usepackage[noend]{algpseudocode}


\usepackage{listings}
\usepackage{color}

\definecolor{gray}{rgb}{0.5,0.5,0.5}
\definecolor{mygreen}{RGB}{66, 189, 107}

\lstset{frame=tb,
  language=Python,
  aboveskip=3mm,
  belowskip=3mm,
  showstringspaces=false,
  columns=flexible,
  basicstyle={\small\ttfamily},
  numbers=none,
  numberstyle=\tiny\color{gray},
  keywordstyle=\color{blue},
  commentstyle=\color{mygreen},
  breaklines=true,
  texcl=true,
  breakatwhitespace=true,
  tabsize=3
}

\floatname{algorithm}{Алгоритм}
\algrenewcommand\algorithmicif{\textbf{Если}}
\algrenewcommand\algorithmicelse{\textbf{иначе}}
\algrenewcommand\algorithmicthen{\textbf{тогда}}
\algrenewcommand\algorithmicrequire{\textbf{Вход:}}
\algrenewcommand\algorithmicensure{\textbf{Выход:}}
\algrenewcommand\algorithmicreturn{\textbf{Вернуть}}
\algrenewcommand\algorithmicfor{\textbf{Для}}
\algrenewcommand\algorithmicdo{\textbf{сделать}}







\setlist[enumerate]{labelsep=0.5pc, labelindent=1.5pc, labelwidth=1.5pc, font=\sffamily\bfseries}
\setlist[enumerate,1]{label=\arabic*., ref=\arabic*}
\setlist[enumerate,2]{label=\emph{\alph*}),
ref=\theenumi.\emph{\alph*}}
\setlist[enumerate,3]{label=\roman*), ref=\theenumii.\roman*}
\setlist[itemize]{leftmargin=*}
\setlist[itemize,1]{label=$\bullet$}


\setcounter{MaxMatrixCols}{30}
\justifying

\graphicspath{{pictures/}}
\DeclareGraphicsExtensions{.pdf,.png,.jpg}

%\usetheme{Singapore}
%\usetheme{boxes}

\usetheme{Madrid}
\usecolortheme{default}
\defbeamertemplate*{footline}{shadow theme}
{%
  \leavevmode%
  \hbox{%
  \begin{beamercolorbox}[wd=.8\paperwidth,ht=2.5ex,dp=1.125ex,,center]{title in head/foot}%
    \insertshorttitle\hfill
  \end{beamercolorbox}%
  \begin{beamercolorbox}[wd=.2\paperwidth,ht=2.5ex,dp=1ex,center]{author in head/foot}%
    \insertframenumber\,/\,\inserttotalframenumber\hfill
  \end{beamercolorbox}}%
}

\newcommand{\jj}{\righthyphenmin=20 \justifying}
\newcommand{\setpsi}[1]{\underset{\ket{\psi_{#1}}}{\uparrow}}

%\newtheorem{proposition}[theorem]{Proposition}
%\newtheorem{corollary}[theorem]{Corollary}
\newtheorem*{remark}{Замечание}
\newtheorem*{property}{Свойство}
\newtheorem{proposition}{Предложение}
\newtheorem{question}{Вопрос}
%\theoremstyle{definition}
%\newtheorem{conjecture}{Гипотеза}



\usepackage{environ}         % provides \BODY
\usepackage{etoolbox}        % provides \ifdimcomp
\usepackage{graphicx}        % provides \resizebox

\newlength{\myl}
\let\origequation=\equation
\let\origendequation=\endequation

\RenewEnviron{equation}{
  \settowidth{\myl}{$\BODY$}                       % calculate width and save as \myl
  \origequation
  \ifdimcomp{\the\linewidth}{>}{\the\myl}
  {\ensuremath{\BODY}}                             % True
  {\resizebox{\linewidth}{!}{\ensuremath{\BODY}}}  % False
  \origendequation
}






\deftranslation[to=russian]{Theorem}{Теорема}
\deftranslation[to=russian]{Definition}{Определение}
\deftranslation[to=russian]{Lemma}{Лемма}\deftranslation[to=russian]{Corollary}{Следствие}

\title{Разработка эвристических методов построения подстановок с низкой дифф. равномерностью}
\author{\and}
\institute{Летняя школа-конференция <<Криптография и информационная безопасность>>}
\date[]{\today}
\centering

\begin{document}
%\maketitle
\justifying
\begin{frame}
    \titlepage
\end{frame}

\begin{frame}{Введение}
\begin{itemize}
    \item Актуальность симметричных шифров
    \item Важность нахождения S- и P-блоков с нужными качествами для обеспечения криптографической стойкости.
\end{itemize}
\end{frame}

\begin{frame}{Основные определения}

\begin{block}{Определение}
    Для $n \in \mathbb{N}$, отображение вида $\mathbb{F}_{2}^{n} \xrightarrow{} \mathbb{F}_{2}$, где $\mathbb{F}_{2}$ --- поле из двух элементов, называется \textit{булевой функцией}.
    Отображения вида $\mathbb{F}_{2}^{n} \xrightarrow{} \mathbb{F}_{2}^{m}$, где $m \in \mathbb{N}$, называются \textit{векторными булевыми функциями} (или $(n, m)$-\textit{функциями}).
\end{block}
$(n,m)-$функцию $F(x_1,...,x_n)$ можно задать $m$ булевыми функциями от $n$ переменных:
$$F(x_1, \dots ,x_n) = (f_1(x_1\dots x_n),\dots,f_m(x_1,\dots, x_n)).$$
Функции $f_i$ называются \textit{координатными функциями}, а произвольная ненулевая линейная комбинация координатных функций называется \textit{компонентной функцией}.
\end{frame}



\begin{frame}{Криптографические характеристики функций}
\begin{block}{Определение}
Алгебраической степенью булевой функции называется степень ее полинома Жегалкина.
\end{block}
\begin{block}{Определение}
Нелинейностью булевой функции $f$ от $n$ переменных называется величина $N_f$, равная расстоянию Хэмминга от $f$ до множества $\mathcal{A}_n$ всех аффинных функций от $n$ переменных.
\end{block}

\begin{block}{Определение}
Алгебраической иммунностью $AI(f)$ функции $f$ называется такое наименьшее число $d$, что существует аннулатор $g$ степени $d$, не тождественно равный нулю, либо для функции $f$, либо для $f \oplus 1$
\end{block}
\end{frame}

\begin{frame}{Криптографические характеристики функций}
Криптографические характеристики булевых функций можно перенести на векторный случай
   \begin{block}{Определение}
Минимальной алгебраической степенью векторной булевой функции называется наименьшая из алгебраических степеней ее компонентных функций.
\end{block} 
\begin{block}{Определение}
Нелинейностью векторной булевой функции называется наименьшая из нелинейностей ее компонентных функций.
\end{block} 
\begin{block}{Определение}
Алгебраической иммуностью векторной булевой функции называется наименьшая из алгебраических иммуностей ее компонентных функций.
\end{block} 
\end{frame}

\begin{frame}{Криптографические характеристики функций}
\begin{block}{Определение}
Векторная булева $(n,m)$-функция $f$ называется \textit{дифференциально $\delta$-равномерной}, если для любых $a \neq 0, b \in \mathbb{F}_{2}^{n}$ уравнение $$f(x) + f(x + a) = b$$ имеет не более $\delta$ решений в $\mathbb{F}_{2}^{n}$. Наименьшее такое число $\delta$ называется \textit{показателем дифференциальной равномерности}.
\end{block}

\begin{block}{Определение}
    Отображение $$f: \mathbb{F}_{2}^{n} \xrightarrow{} \mathbb{F}_{2}^{n}$$
    называется \textit{APN-отображением}, если оно дифференциально $2$-равномерно.
\end{block}


\end{frame}




\begin{frame}{Комбинаторное описание дифференциальной равномерности}
\begin{tikzpicture}

    \node (Q1) at (0,0) {$S(\mathbb{F}_{2^n})$};
    \node (Q2) at (1,2) {$f(x) = \sum\limits_{i=0}^n a_ix^i$};
    \node (Q3) at (0,4) {APN-отображения};
    \node (Q4) at (-3,2) {$F(x_1,...,x_n) = (f_1(x),...,f_n(x))$};

    \draw (Q1)--(Q2);
    \draw (Q1)--(Q4);
    \draw (Q3)--(Q4);
    \draw (Q2)--(Q3);
    \draw [dashed,red, out=-1,in=0,distance=3cm](Q1) to node [pos=0.5, right,inner sep=0.25cm] {?}(Q3);

    \end{tikzpicture}
\end{frame}






\begin{frame}{Комбинаторное описание дифференциальной равномерности}
Рассмотрим симметрическую группу $S(\Omega)$ на множестве $\Omega$ из $n$ элементов.
\begin{itemize}
\item Расстоянием между подстановками $f,g \in S(\Omega)$  называется величина $$d(f,g) = |\{ x \in \Omega \; | \; f(x) \neq g(x)\} |$$

\item Расстоянием между подгруппами $G,G' \leq S(\Omega)$ назовем $$d(G,G') = \min_{\substack{g \in G \setminus \{e\}  \\ g' \in G' \setminus \{e\}  }} d(g ,g')$$
\end{itemize}
\end{frame}








\begin{frame}{Комбинаторное описание дифференциальной равномерности}
\begin{itemize}
\item Далее будем рассматривать $\Omega = \mathbb{F}_{2^n}$
\item Любой элемент поля $\alpha \in \mathbb{F}_{2^n}$ определяет биективное отображение 
\begin{alignat*}{2}
\tau_{\alpha} \colon \mathbb{F}_{2^n}&\rightarrow {}&& \mathbb{F}_{2^n} \\
           x&\mapsto     {}&& x + \alpha \\        
\end{alignat*}
\item Множество таких отображений $T = \{ \tau_{ \alpha } \; | \; \alpha \in \mathbb{F}_{2^n} \}$ образует подгруппу симметрической группы $S(\mathbb{F}_{2^n})$
\end{itemize}
\end{frame}




\begin{frame}{Комбинаторное описание дифференциальной равномерности}
В работе использовалась следующая комбинаторная характеризация дифференциальной равномерности:
\begin{block}{Утверждение}
Пусть $f \in S(\mathbb{F}_{2^n})$, $T$ -- группа сдвигов, определенная выше, а $$G = f^{-1} \cdot T \cdot f = \{f^{-1} \cdot t \cdot f | t \in T \}.$$ Тогда подстановка f является дифференциально $\delta$-равномерной $ \iff d(G,T) = 2^n-\delta$
\end{block}

\end{frame}



\begin{frame}{Комбинаторное описание дифференциальной равномерности}
    $$\pi = (2,5)(3,6)(4,7) \in \mathbb{F}_{2^3}$$
\begin{table}[h]
\centering
\begin{tabular}{@{}c@{\hspace{1em}}c@{\hspace{1em}}c@{}}

 $T$ & \multicolumn{1}{c}{\stackunder{$\rightarrow$}{\footnotesize $\pi^{-1} \cdot T \cdot \pi$}} &  $G$ \\

$(0,1)(2,3)(4,5)(6,7)$ & & $(0,1)(2,7)(3,4)(5,6)$ \\
$(0,2)(1,3)(4,6)(5,7)$ &  & $(0,5)(1,6)(2,4)(3,7)$ \\
$(0,3)(1,2)(4,7)(5,6)$ & & $(0,6)(1,5)(2,3)(4,7)$ \\
$(0,4)(1,5)(2,6)(3,7)$ & & $(0,7)(1,2)(3,5)(4,6)$ \\
$(0,5)(1,4)(2,7)(3,6)$ &  & $(0,2)(1,7)(3,6)(4,5)$ \\
$(0,6)(1,7)(2,4)(3,5)$ &  & $(0,3)(1,4)(2,6)(5,7)$ \\
$(0,7)(1,6)(2,5)(3,4)$ & & $(0,4)(1,3)(2,5)(6,7)$ \\

\end{tabular}
\end{table}
\end{frame}


\begin{frame}{Комбинаторное описание дифференциальной равномерности}
    $$\sigma = \pi \cdot \textcolor{red}{(2,3)} = (2,5)(3,6)(4,7)\textcolor{red}{(2,3)} = \textcolor{red}{(2,5,3,6)(4,7)} \in \mathbb{F}_{2^3}$$
    $$\sigma^{-1} \cdot T \cdot \sigma = \textcolor{red}{(2,3)}\cdot(\pi^{-1} \cdot T \cdot \pi)\cdot\textcolor{red}{(2,3)}=\textcolor{red}{(2,3)}\cdot G \cdot \textcolor{red}{(2,3)}$$
\begin{table}[h]
\centering
\begin{tabular}{@{}c@{\hspace{1em}}c@{\hspace{1em}}c@{}}

 $G$ & \multicolumn{1}{c}{\stackunder{$\rightarrow$}{\footnotesize $\textcolor{red}{(2,3)} \cdot G \cdot \textcolor{red}{(2,3)}$}} &  $G'$ \\
 

$(0,1)(\textcolor{red}{2},7)(\textcolor{red}{3},4)(5,6)$& &  $(0,1)(\textcolor{red}{2},4)(\textcolor{red}{3},7)(5,6)$\\
$(0,5)(1,6)(\textcolor{red}{2},4)(\textcolor{red}{3},7)$& &  $(0,5)(1,6)(\textcolor{red}{2},7)(\textcolor{red}{3},4)$\\
$(0,6)(1,5)(\textcolor{red}{2},\textcolor{red}{3})(4,7)$& &  $(0,6)(1,5)(\textcolor{red}{2},\textcolor{red}{3})(4,7)$\\
$(0,7)(1,\textcolor{red}{2})(\textcolor{red}{3},5)(4,6)$& &  $(0,7)(1,\textcolor{red}{3})(\textcolor{red}{2},5)(4,6)$\\
$(0,\textcolor{red}{2})(1,7)(\textcolor{red}{3},6)(4,5)$& &  $(0,\textcolor{red}{3})(1,7)(\textcolor{red}{2},6)(4,5)$\\
$(0,\textcolor{red}{3})(1,4)(\textcolor{red}{2},6)(5,7)$& &  $(0,\textcolor{red}{2})(1,4)(\textcolor{red}{3},6)(5,7)$\\
$(0,4)(1,\textcolor{red}{3})(\textcolor{red}{2},5)(6,7)$& &  $(0,4)(1,\textcolor{red}{2})(\textcolor{red}{3},5)(6,7)$\\

\end{tabular}
\end{table}
\end{frame}




\begin{frame}{Задачи}
\begin{itemize}
    \item Разработать эффективный алгоритм вычисления дифференциальной равномерности при умножении подстановки на транспозицию
    \item Разработать алгоритм построения подстановок с низкой дифференциальной равномерностью
\end{itemize}
\end{frame}



\begin{frame}{Алгоритм вычисления дифференциальной равномерности после умножения на транспозицию}
\begin{itemize}
    \item Рассмотрим $\pi \in S(\mathbb{F}_{2^n})$ и $G = \pi^{-1} \cdot T \cdot \pi$.
    \item Пусть $N = |\mathbb{F}_{2^{n}}|$, $G = \{g_{1}, \dots, g_{N - 1}\}$, $T = \{t_{1}, \dots, t_{N - 1}\}$.
    \item Каждый элемент $t_i$ раскладывается в произведение независимых транспозиций $t_i=\tau_1^i \dots \tau_{\frac{N}{2}}^i$.
    \item Зафиксируем некоторый элемент $g = \sigma_{1}\dots\sigma_{\frac{N}{2}} \in G$.
    \item Определим множество $I(t_{i}, g) \stackrel{\text{def}}{:=} t_{i} \cap g$.
\end{itemize}

\begin{block}{Утверждение}
 Сложность построения множества $I(t_i,g)$ равна $O(N)$ и  $d(t_{i}, g) = N - 2 |I(t_{i}, g)|$
\end{block}




\end{frame}

\begin{frame}{Алгоритм вычисления дифференциальной равномерности после умножения на транспозицию}
\begin{itemize}
    \item Определим
$$I(g) \stackrel{\text{def}}{:=} \{I(t_{i}, g) \text{ } | \text{ } t_{i} \in T \},$$
$$D(g) \stackrel{\text{def}}{:=} (I(g), \underset{i \in I(g)}{\max} {\{2 \cdot|i| \}})$$
\end{itemize}

\begin{block}{Утверждение}
 Сложность построения $D(g)$ равна $O(N^{2})$.
\end{block}
\end{frame}

\begin{frame}{Алгоритм вычисления дифференциальной равномерности после умножения на транспозицию}
\begin{itemize}
    \item Определим множество $\Delta(G) \stackrel{\text{def}}{:=} \{D(g) \text{ } | \text{ } g \in G\}$. 
\end{itemize}

\begin{block}{Утверждение}
 Дифференциальная равномерность подстановки $\pi$ равна $$\delta = \underset{(d_{1}, d_{2}) \in \Delta(G)}{\max} \{d_{2} \}.$$ Сложность вычисления $\delta$ равна $O(N^{3})$.
\end{block}


Сложность классического алгоритма $O(N^2)$...
\end{frame}

\begin{frame}{Алгоритм вычисления дифференциальной равномерности после умножения на транспозицию}

Пусть $I(t_i,g)$ уже построено и $\sigma = (\alpha \beta)$ -- некоторая транспозиция.

Тогда вычислить $I(t_i,\sigma^{-1}g\sigma)$ можно следующим образом:
\begin{itemize}
\item Если $\sigma = \sigma_i$, то $I(t_i,g)=I(t_i,\sigma^{-1}g\sigma)$
\item Пусть $\sigma \neq \sigma_i$. Тогда $\exists \sigma_s=(\alpha, \alpha'),\sigma_r=(\beta, \beta')$ в разложении $g$. Тогда $$\sigma^{-1}g\sigma = \sigma_1...\sigma_s'...\sigma_r'...\sigma_{\frac{N}{2}}, \text{ где } \sigma_s'=(\beta, \alpha'),\sigma_r'=(\alpha, \beta').$$
Таким образом $I(t_i, \sigma^{-1}g\sigma)$ получается из $I(t_i,g)$ удалением $\sigma_s,\sigma_r$ (если они там есть) и добавлением $\sigma_s',\sigma_r'$ (при условии, что они есть в разложении $t_i$)
\end{itemize}

\begin{block}{Утверждение}
 Сложность вычисления $I(t_i,\sigma^{-1}g\sigma)$ по известному $I(t_i,g)$ равна $O(1)$.
\end{block}
\end{frame}


\begin{frame}{Алгоритм вычисления дифференциальной равномерности после умножения на транспозицию}

Для вычисления множества $D(\sigma^{-1}g\sigma)$ по $D(g) = (I(g), d)$ нужно:
\begin{itemize}
\item Если $\sigma = \sigma_i$, то $D(g)$ не изменится
\item Найти сдвиги $t(\sigma_s), t(\sigma_r), t(\sigma_s'), t(\sigma_r')$ и пересчитать $I(t(\sigma_s),g)$, $I(t(\sigma_r),g)$, $I( t(\sigma_s'),g)$, $I(t(\sigma_r'),g)$
\item Обновить максимальное значение $d$.
\end{itemize}

\begin{block}{Утверждение}
 Сложность вычисления $D(\sigma^{-1}g\sigma)$ по известному $D(g)$ равна $O(1)$.
\end{block}

\begin{block}{Утверждение}
 Сложность вычисления $\Delta(\sigma^{-1} G  \sigma)$ по известному $\Delta(G)$ равна $O(N)$.
\end{block}
\end{frame}





\begin{frame}{Подходы к построению подстановок с низкой дифф. равномерностью}

Наиболее удачные подходы:

\begin{itemize}
\item Полный перебор транспозиций.
\item Перебор образующих транспозиций.
\item Перебор транспозиций, образованных элементами пересечения.
\end{itemize}

\end{frame}

%------------------------------------------------------------




\begin{frame}{``Комбинированный'' подход}

После многочисленных экспериментов оказалось, что лучше всего справляется с задачей уменьшения дифференциальной равномерности ``комбинированный'' подход, который состоит из поочерёдного применения подходов с перебором образующих транспозиций и с перебором транспозиций, образованных элементами пересечения. Такой ``комбинированный'' подход позволяет стабильно снижать дифференциальную равномерность S-блока размерности 8 до значения 6.

\end{frame}

%------------------------------------------------------------


\begin{frame}{Результаты}
Используя построенные алгоритмы, проводились вычислительные эксперименты. Была построена подстановка на $\mathbb{F}_{2^8}$:
$$\scriptscriptstyle \pi = (1, 17, 29, 209, 85, 91, 54, 34, 95, 157, 255, 147, 128, 25, 23, 75, 113, 72, 87, 156, 204, 102, 130,227, 121, 44, 19, 151, $$
$$\scriptscriptstyle137, 123, 243, 237, 165, 61, 27, 183, 142, 88, 70, 191, 163, 170, 99, 20, 103, 146, 166, 139, 111, 182, 31, 59, 154, 116, 150,$$
$$\scriptscriptstyle 160, 196, 7, 205, 48, 21, 241, 153, 112, 178, 189, 93, 198, 253, 201, 3, 225, 5, 172, 155, 242, 175, 176, 184, 108, 122, 28, 190,$$
$$\scriptscriptstyle 77, 194, 42, 132, 129, 148, 52, 79, 131, 152, 246,05, 186, 109, 219, 248, 134, 9, 35, 217, 211, 64, 149, 249, 222, 164, 226, 30, $$
$$\scriptscriptstyle41, 74, 229, 117, 254, 221, 199,145, 15, 81, 218, 247, 212, 51, 47, 119, 144, 200, 159, 60, 181, 173, 65, 4, 8, 236, 104, 208, 43, $$
$$\scriptscriptstyle71, 233, 136,50, 115, 207, 140, 63, 107, 6, 86, 214, 203, 216, 223, 101, 228, 49, 67, 239, 114, 138, 83, 53, 13, 124, 180, 96,234,$$
$$\scriptscriptstyle 22, 231, 210, 16, 26, 125, 82, 100, 206, 18, 94, 24, 169, 106, 69, 202, 126, 11, 141, 12, 250, 133, 193, 33,251, 92, 98, 174, 185,$$
$$\scriptscriptstyle 252, 90, 158, 197, 188, 118, 127, 80, 57, 215, 135, 45, 56, 162, 10, 195, 36, 62, 177, 245,213, 58, 110, 192, 2, 40, 167, 168, 143,$$
$$\scriptscriptstyle 38, 66, 187, 39, 240, 68, 220, 32, 120, 76, 55, 97, 161, 235, 46, 37, 14,78, 89, 73, 238, 171, 232, 230, 179, 84, 224, 244, 0).$$

\end{frame}

\begin{frame}{Результаты}
\begin{table}[h!]
\centering
\caption{Таблица 1}
\label{tab:permutation_comparision}
\begin{tabular}{|m{4cm}|c|c|c|c|} % Используем m-колонки для принудительного многострочного режима
\hline
&\textbf{$\pi_{AES}$}&\textbf{$\pi_{Kuznechik}$}&\textbf{$\pi_{Jimenez}$\footnotemark[1]}&\textbf{$\pi$}\\ \hline
\makecell[c]{Дифференциальная\\равномерность}&4&8&6&6\\\hline
\makecell[c]{Нелинейность}&112&100&104&100\\\hline
\makecell[c]{Минимальная\\алгебраическая\\степень}&7&7&7&7\\\hline
\makecell[c]{Алгебраическая\\иммунность}&4&4&4&4\\\hline
\end{tabular}
\end{table}

\footnotetext[1]{R. A. de la Cruz Jimenez, “Constructing 8-bit permutations, 8-bit involutions and 8-bit orthomorphisms with almost optimal cryptographic parameters”, Матем. вопр. криптогр., 12:3 (2021), 89–124}
\end{frame}

\begin{frame}{Направления дальнейшего исследования}

\begin{itemize}
\item Придумать и опробовать новые эвристики.
\item Переписать код на более ``быстрый'' язык.
\item Использовать более мощный компьютер для перебора.
\item Изучить возможность <<эффективного>> построения группы G находящихся на большом расстоянии от $T$
\item Изучить возможность описания <<легко проверяемого>> необходимого условия существования групп G с расстоянием $2^n-2$
\end{itemize}

\end{frame}


\begin{frame}
\huge{\centerline{Спасибо за внимание!}}
\end{frame}
\end{document}