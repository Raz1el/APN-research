\documentclass[a4paper,12pt]{article}
\usepackage[english,russian]{babel}
\usepackage[utf8x]{inputenc}
\usepackage{latexsym,mathrsfs}
\usepackage{stmaryrd, enumitem}
\usepackage{amsthm,amsfonts,amssymb,amsmath}
\usepackage{geometry}
\usepackage{tempora}
\usepackage[pdftex]{graphicx}
\usepackage{comment}
\usepackage{xcolor}

% \usepackage{pythonhighlight}
\usepackage{listings}
\usepackage{color}

\definecolor{gray}{rgb}{0.5,0.5,0.5}

\lstset{frame=tb,
  language=Python,
  aboveskip=3mm,
  belowskip=3mm,
  showstringspaces=false,
  columns=flexible,
  basicstyle={\small\ttfamily},
  numbers=none,
  numberstyle=\tiny\color{gray},
  keywordstyle=\color{blue},
  breaklines=true,
  breakatwhitespace=true,
  tabsize=3
}

\geometry{top=17mm}  \geometry{bottom=20mm}
\geometry{left=17mm} \geometry{right=17mm}
\linespread{1.1}
\parindent=5mm


\def\titleK#1{\begin{center}{\textbf {#1}}\end{center}}
\def\authorK#1{\begin{center}{#1}\end{center}}



\newenvironment{abstractK}{}



\newtheorem{lemma}{Лемма}
\newtheorem{statement}{Утверждение}
\theoremstyle{definition}
\newtheorem{theorem}{Теорема}
\newtheorem{corollary}{Следствие}
\newtheorem{proposition}{Предложение}
\theoremstyle{definition}
\newtheorem{definition}{Определение}
\theoremstyle{definition}
\newtheorem{question}{Вопрос}
\theoremstyle{definition}
\newtheorem{conjecture}{Гипотеза}

%ВАЖНО: Не менять и не добавлять ничего выше этой строки.
%Изменения вносить только внутри окружения \begin{document}\end{document}

\usepackage{algorithm}
\usepackage{algpseudocode}

\begin{document}

%Название доклада
\titleK{Разработка эвристических методов построения подстановок с низкой дифференциальной равномерностью}
%Информация об авторах
\authorK{А.Р. Белов$^{1}$, В.М. Завьялова$^{2}$, Т.А. Хаирнуров$^{1}$, Д.Г. Юргенсон$^{1}$\\
$^{1}$ЯрГУ им. П.Г. Демидова \\$^{2}$СПГУ им. Екатерины II \\
{\bf E-mail:} ashmedey@gmail.com, z.varvaramaksimovna@gmail.com, t.hairnurov@uniyar.ac.ru, d.yurgenson@uniyar.ac.ru}
%Текст тезисов доклада 
%Внутри текста можно использовать стандартные команды TeX а также следующие определенные окружения
%Теорема - \begin{theorem}\end{theorem}
%Лемма - \begin{lemma}\end{lemma}
%Следствие - \begin{corollary}\end{corollary}
%Предложение - \begin{proposition}\end{proposition}
%Определение - \begin{definition}\end{definition}
%Вопрос - \begin{question}\end{question}
%Гипотеза - \begin{conjecture}\end{conjecture}
\begin{abstract}
      Изучается дифференциальная равномерность подстановок на конечном поле $\mathbb{F}_{2^n}$. На основе комбинаторного представления порядка дифференциальной равномерности подстановки предложен эффективный алгоритм вычисления дифференциальной равномерности после умножения на транспозицию. Рассмотрены некоторые подходы к построению подстановок с низкой дифференциальной равномерностью.\\
\\{\bf Ключевые слова:} \textit{ дифференциальная равномерность, подстановка, симметрическая группа, расстояние Хэмминга.} 
 \end{abstract}


\begin{abstractK}


%\textit{Задачу опишите полнее. Введя те ограничения, о которых мы говорили, включая количество узлов.}


%{=====================================================================================================================================================================================================================================}

\section*{Введение}

Симметричные шифры уже много лет являются одной из самых важных составляющих криптографии, и, как кажется, они не потеряют своей значимости через 5, 10 и даже 15 лет, ведь появление квантового компьютера не столь критично отразится на симметричной криптографии, в отличие от асимметричной. В связи с этим создание новых и улучшение уже существующих симметричных шифров крайне актуально. Популярной основой для симметричных шифров являются SP-сети. К примеру, некоторые современные стандарты шифрования (``Кузнечик'', AES) построены именно на SP-сетях \cite{LOS}. Как следует из самого названия, SP-сеть в простейшем варианте представляет собой многократно используемые по очереди слои двух типов: подстановочный слой (S-слой) и перестановочный слой (P-слой), которые, в свою очередь, состоят из S-блоков и P-блоков соответственно. Именно от характеристик этих двух слоёв во многом зависит общая криптографическая стойкость шифра, поэтому остро стоит вопрос о нахождении таких блоков S и P, которые обладали бы всеми нужными качествами, чтобы шифр, в состав которого входят слои, включающие в себя эти блоки, считался криптографически стойким. Один из возможных способов нахождения таких S-блоков детально описан в данной работе.

\section*{Основные определения}

\begin{definition}
    Для $n \in \mathbb{N}$, отображение вида $\mathbb{F}_{2}^{n} \xrightarrow{} \mathbb{F}_{2}$, где $\mathbb{F}_{2}$ --- поле из двух элементов, называется \textit{булевой функцией}.
    Отображения вида $\mathbb{F}_{2}^{n} \xrightarrow{} \mathbb{F}_{2}^{m}$, где $m \in \mathbb{N}$, называются \textit{векторными булевыми функциями} (или $(n, m)$-\textit{функциями}).
\end{definition}

Одной из характеристик нелинейных элементов блочных шифров, которая обеспечивает устойчивость к некоторым методам анализа, является \textit{дифференциальная равномерность}. 

\begin{definition}
    Векторная булева $(n,m)$-функция $f$ называется \textit{дифференциально $\delta$-равномерной}, если для любых $a \neq 0, b \in \mathbb{F}_{2^n}$ уравнение $$f(x) + f(x + a) = b$$ имеет не более $\delta$ решений в $\mathbb{F}_{2}^{n}$. Наименьшее такое число $\delta$ называется \textit{показателем дифференциальной равномерности}.
\end{definition}

Отображения, обладающие оптимальной дифференциальной равномерностью, называются \textit{почти совершенно нелинейными отображениями} или \textit{APN-отображениями}. Далее объекты $\mathbb{F}_2^n$ и $\mathbb{F}_{2^n}$ отождестявляются.

\begin{definition}
    Отображение $$f: \mathbb{F}_{2^n} \xrightarrow{} \mathbb{F}_{2^n}$$
    называется \textit{APN-отображением}, если оно дифференциально $2$-равномерно.
\end{definition}

Вопрос о существовании биективных APN-отображений все еще открыт. Известно, что для полей $\mathbb{F}_{2^n}$ при $n = 2, 4$ таких отображений не существует. В работе \cite{Hou} впервые был построен пример APN-отображения для $n = 6$.

Для характеризации дифференциальной равномерности мы используем понятие \textit{расстояния Хэмминга между подстановками} \cite{Belov}. Пусть $\Omega$ --- множество из $n$ элементов, $S(\Omega)$ --- симметрическая группа на $\Omega$ с операцией произведения подстановок, определенной по правилу $[\pi \cdot \sigma](x) = \sigma(\pi(x))$. Под пересечением подстановок будем понимать множество $$f \cap g = \{\tau | \text{ цикл } \tau \text{ входит в разложение на независимые циклы подстановок }f \text{ и }g\}.$$

\begin{definition}
    \textit{Расстоянием Хэмминга между подстановками} $f, g \in S(\Omega)$ называется $$d(f, g) = |\{ x \in \Omega: f(x) \neq g(x) \}|$$
\end{definition}

\begin{definition}
    \textit{Расстоянием Хэмминга между подгруппами} $G, G' \leq S(\Omega)$ называется $$d(G, G') = \underset{\substack{g \in G \setminus \{ e \} \\ g' \in G' \setminus \{ e \}}}{\min} d(g, g')$$
\end{definition}

Для для вычисления расстояния между подстановками по их разложению в произведение независимых транспозиций используется

\begin{statement}
Пусть разложение в произведение независимых циклов $f,g \in S(\Omega)$ имеет вид: $$f = (x_1, y_1) ... (x_s, y_s) \tau_1 ... \tau_k,$$ $$ g=(x_1,y_1) ... (x_s, y_s) \sigma_1 ... \sigma_l,$$  где  $\tau_i, \sigma_j$  различные транспозиции.\\
Тогда $$d(f,g) = n - 2s - |fix(f) \cap fix(g)|,$$ 
где $$fix(\pi) = \{ x \in \Omega \; | \; \pi(x) = x \}.$$
\end{statement}

Далее будем рассматривать $\Omega = \mathbb{F}_{2^n}$. Любой элемент поля $\alpha \in \mathbb{F}_{2^n}$ определяет биективное отображение 
\begin{alignat*}{2}
\tau_{\alpha} \colon \mathbb{F}_{2^n}&\rightarrow {}&& \mathbb{F}_{2^n} \\
           x&\mapsto     {}&& x + \alpha \\        
\end{alignat*}
Множество таких отображений $T = \{ \tau_{ \alpha } \; | \; \alpha \in \mathbb{F}_{2^n} \}$ образует подгруппу симметрической группы $S(\mathbb{F}_{2^n})$

Характеризацию дифференциальной равномерности подстановки дает

\begin{statement}

    Пусть $f \in S(\mathbb{F}_{2^n})$, $T$ -- группа сдвигов, определенная выше, а $$G = f^{-1} \cdot T \cdot f = \{f^{-1} \cdot t \cdot f | t \in T \}.$$ Тогда подстановка f является дифференциально $\delta$-равномерной $ \iff d(G,T) = 2^n-\delta$
\end{statement}
Пусть подстановка $f'$ получена из $f$ умножением справа на транспозицию $\tau$. Тогда группа сдвигов сопряженная $f'$ может быть выражена $$f'^{-1}\cdot T \cdot f' = (f\cdot\tau)^{-1} \cdot T \cdot (f\cdot\tau) = \tau^{-1} \cdot (f^{-1} \cdot T \cdot f) \cdot \tau = \tau^{-1} \cdot G \cdot \tau$$
Известно, что при умножении подстановки на транспозицию показатель дифференциальной равномерности изменяется не более чем на 4

\begin{statement}
    Пусть $d(G, T) = \alpha$. Тогда, если $\tau$ -- транспозиция, то
$$ \alpha - 4 \leq d(\tau^{-1} \cdot G \cdot \tau, T) \leq \alpha + 4 $$
\end{statement}

\section*{Алгоритм вычисления дифференциальной равномерности}

Рассмотрим $\pi \in S(\mathbb{F}_2^n)$ и $G = \pi^{-1} \cdot T \cdot \pi$. Пусть $N = |\mathbb{F}_{2^{n}}|$, $G = \{g_{1}, \dots, g_{N - 1}\}$, $T = \{t_{1}, \dots, t_{N - 1}\}$. Каждый элемент $t_i$ раскладывается в произведение независимых транспозиций $t_i=\tau_1^i \dots \tau_{\frac{N}{2}}^i$. Зафиксируем некоторый элемент $g = \sigma_{1}\dots\sigma_{\frac{N}{2}} \in G$ и
определим множество $I(t_{i}, g) \stackrel{\text{def}}{:=} t_{i} \cap g$.
Тогда $d(t_{i}, g) = N - 2 |I(t_{i}, g)|$.

\begin{statement}
    Сложность построения $I(t_{i}, g)$ равна $O(N)$.
\end{statement}

Определим
$I(g) \stackrel{\text{def}}{:=} \{I(t_{i}, g) \text{ } | \text{ } t_{i} \in T \},
D(g) \stackrel{\text{def}}{:=} (I(g), \underset{i \in I(g)}{\max} {\{2 \cdot|i| \}})$

\begin{statement}
    Сложность построения $D(g)$ равна $O(N^{2})$.
\end{statement}
Определим множество $\Delta(G) \stackrel{\text{def}}{:=} \{D(g) \text{ } | \text{ } g \in G\}$. 
\begin{statement}
    Дифференциальная равномерность подстановки $\pi$ равна $$\delta = \underset{(d_{1}, d_{2}) \in \Delta(G)}{\max} \{d_{2} \}.$$
\end{statement}
\begin{statement}
    Сложность вычисления $\delta$ равна $O(N^{3})$.
\end{statement}

Пусть $\Delta(G)$ уже вычислено и $\pi' = \pi \cdot (\alpha, \beta)$. Покажем как вычислить $\Delta((\alpha, \beta) \cdot G \cdot (\alpha, \beta))$ 

Рассмотрим изменение множества $I(t_i, g)$ под действием транспозиции $\sigma=(\alpha, \beta)$. Напомним, что $g = \sigma_{1}\dots\sigma_{\frac{N}{2}}$. Если $\sigma=\sigma_i$, то $I(t_i, g)=I(t_i, \sigma^{-1} g \sigma)$. Если $\sigma \neq \sigma_i$, тогда в разложении $g$ на транспозиции присутствуют транспозиции $\sigma_s=(\alpha, \alpha ')$ и $\sigma_r=(\beta, \beta ')$. В таком случае, под действием транспозиции $\sigma$ из разложения $g$ пропадут транспозиции $\sigma_s$ и $\sigma_r$, но появятся новые транспозиции $\sigma_s ' =(\beta, \alpha ')$ и $\sigma_r ' =(\alpha, \beta ')$. Отсюда следует, что $I(t_i, \sigma^{-1} g \sigma)$ получается из $I(t_i, g)$ удалением $\sigma_s$ и $\sigma_r$ (если таковые имеются) и добавлением $\sigma_s '$ и $\sigma_r '$ (если они есть в разложении $t_i$). При правильном выборе структур данных, транспозиции $\sigma_s$, $\sigma_r$ можно находить за $O(1)$. Определять принадлежность транспозиций $\sigma_s'$, $\sigma_r'$ разложению $t_i$ также можно за $O(1)$. Таким образом, итоговая сложность вычисления $I(t_i, \sigma^{-1} g \sigma)$ равна  $O(1)$.

Аналогичным образом описывается процедура нахождения множества $D(\sigma^{-1} g \sigma)$ по известному $D(g) = (I(g), d)$. Если $\sigma = \sigma_{i}$, то $D(g)$ не меняется. Иначе нужно найти сдвиги $t(\sigma_{s}), t(\sigma_{r}), t(\sigma'_{s}), t(\sigma'_{r})$, в разложение которых входят $\sigma_{s}$,$\sigma_{r}$,$\sigma_{s}'$ и $\sigma_{r}'$, а затем пересчитать множества $$I(t(\sigma_{s}), g), I(t(\sigma_{r}), g), I(t(\sigma'_{s}), g), I(t(\sigma'_{r}), g).$$ Если учесть, что мощность этих множеств не может измениться более чем на 2, то сложность нахождения нового максимума $d' = \underset{i \in I(\sigma^{-1} g \sigma)}{\max} {\{2 \cdot|i| \}}$ можно обеспечить за время $O(1)$.

Подытоживая, результирующая сложность вычисления $\Delta(\sigma^{-1} G \sigma)$ равна $O(N)$. Если учесть, что $\delta$ изменяется не более чем на 4, то $\delta' = \underset{(d_{1}, d_{2}) \in \Delta(\sigma^{-1} G \sigma)}{\max} \{d_{2} \}$ вычисляется за $O(1)$.

\section*{Некоторые подходы к построению подстановок с низкой дифференциальной равномерностью}

В ходе работы было предложено несколько подходов, позволяющих уменьшить дифференциальную равномерность подстановки путём её умножения на n-ое количество транспозиций. Основная идея всех подходов заключается в следующем: необходимо выбрать какую-либо подстановку, далее следует найти такую транспозицию, при умножении на которую либо уменьшится дифференциальная равномерность подстановки, либо уменьшится метрика $\chi(\Delta(G))$, напрямую связанная с дифференциальной равномерностью. Затем нужно сопрячь подгруппу данной транспозицией, если таковая нашлась, и повторить шаг с подбором транспозиции или заврешить поиск, если подходящую транспозицию найти не удалось. Вот наиболее удачные алгоритмы, основанные на этой идее:

\begin{itemize}
\item[1.] Полный перебор всех возможных транспозиций. Этот алгоритм даёт лучшие результаты, но при этом он работает очень медленно на больших размерностях.
\item[2.] Перебор некоторого набора транспозиций. В качестве возможных наборов нами были выбраны различные системы образующих симметрических групп.
\item[3.] Перебор транспозиций из множества $Tr(\Delta(G))$, построенного с помощью эвристики.
\end{itemize}

Необходимо пояснить, что представляют из себя метрика $\chi(\Delta(G))$ и множество $Tr(\Delta(G))$. Но для начала определим другое множество:

$$M_{d} \stackrel{\text{def}}{:=} \{ t \cap g \text{ } | \text{ } |I(t, g)| = \frac{d}{2} \}$$

Теперь, с помощью введённого множества, определим $\chi(\Delta(G))$ и $Tr(\Delta(G))$:

$$\chi(\Delta(G)) \stackrel{\text{def}}{:=} |M_{\delta}|$$

$$Tr(\Delta(G)) \stackrel{\text{def}}{:=} \{ (\alpha, \beta) \text{ } | \text{ } \alpha \in \tau_1 \text{ } \& \text{ } \beta \in \tau_2 \text{, где } \tau_1, \tau_2 \in \cup M_{\delta} \text{ различные транспозиции} \}$$

Обозначим как $Transpositions_{1}$ - множество всех возможных транспозиций, $Transpositions_{2}$ - множество образующих транспозиций, $Transpositions_{3}$ - множество $Tr(\Delta(G))$. Псевдокод первого метода приведен в Алгоритме~\ref{Fig1}. Псевдокод для алгоритма 2 (3) отличается лишь заменой $Transpositions_{1}$ на $Transpositions_{2}$ ($Transpositions_{3}$).\\

\begin{algorithm}
\begin{algorithmic}
\Require $\Delta(G)$
\Ensure $\Delta(G')$ \Comment{где $d(G', T) > d(G, T)$ или $d(G', T) = d(G, T)$ \& $\chi(\Delta(G')) < \chi(\Delta(G))$}
\State $\Delta(G') \gets \Delta(G)$
\State $MaxDistance \gets d(G, T)$
\For{$Transposition \in Transpositions_{1}$}
    \State $\Delta(G'') = RecalcGroupDistance(\Delta(G), Transposition)$ \Comment{Сопрягаем $G$ транспозицией}
\If{$d(G'', T) > MaxDistance$}
    \State $\Delta(G') \gets \Delta(G'')$
    \State $MaxDistance \gets d(G', T)$
\EndIf
\If{$d(G'', T) = MaxDistance$ \& $\chi(\Delta(G'')) < \chi(\Delta(G'))$}
    \State $\Delta(G') \gets \Delta(G'')$
\EndIf
\EndFor
\end{algorithmic}
\caption{Алгоритм 1 }
\label{Fig1}
\end{algorithm}

После многочисленных экспериментов оказалось, что лучше всего справляется с задачей уменьшения дифференциальной равномерности ``комбинированный'' алгоритм, который состоит из последовательного применения алгоритма 2 с разными системами образующих и алгоритма 3. Такой ``комбинированный'' алгоритм позволяет стабильно снижать дифференциальную равномерность S-блока размерности 8 до значения 6.

\section*{Результаты вычислительных экспериментов}

Используя построенные алгоритмы, проводились вычислительные эксперименты. Была построена подстановка на $\mathbb{F}_{2^8}:$ $$\pi = (1, 17, 29, 209, 85, 91, 54, 34, 95, 157, 255, 147, 128, 25, 23, 75, 113, 72, 87, 156, 204, 102, 130,$$
$$227, 121, 44, 19, 151, 137, 123, 243, 237, 165, 61, 27, 183, 142, 88, 70, 191, 163, 170, 99, 20, 103, 146, 166,$$ $$139, 111, 182, 31, 59, 154, 116, 150, 160, 196, 7, 205, 48, 21, 241, 153, 112, 178, 189, 93, 198, 253, 201, 3,$$ 
$$225, 5, 172, 155, 242, 175, 176, 184, 108, 122, 28, 190, 77, 194, 42, 132, 129, 148, 52, 79, 131, 152, 246,$$ 
$$105, 186, 109, 219, 248, 134, 9, 35, 217, 211, 64, 149, 249, 222, 164, 226, 30, 41, 74, 229, 117, 254, 221, 199,$$ 
$$145, 15, 81, 218, 247, 212, 51, 47, 119, 144, 200, 159, 60, 181, 173, 65, 4, 8, 236, 104, 208, 43, 71, 233, 136,$$ 
$$50, 115, 207, 140, 63, 107, 6, 86, 214, 203, 216, 223, 101, 228, 49, 67, 239, 114, 138, 83, 53, 13, 124, 180, 96,$$ 
$$234, 22, 231, 210, 16, 26, 125, 82, 100, 206, 18, 94, 24, 169, 106, 69, 202, 126, 11, 141, 12, 250, 133, 193, 33,$$ 
$$251, 92, 98, 174, 185, 252, 90, 158, 197, 188, 118, 127, 80, 57, 215, 135, 45, 56, 162, 10, 195, 36, 62, 177, 245,$$ 
$$213, 58, 110, 192, 2, 40, 167, 168, 143, 38, 66, 187, 39, 240, 68, 220, 32, 120, 76, 55, 97, 161, 235, 46, 37, 14,$$ 
$$78, 89, 73, 238, 171, 232, 230, 179, 84, 224, 244, 0).$$

Сравнительная характеристика с другими известными подстановками приведена в таблице \ref{tab:permutation_comparision}.
В качестве параметров сравнения выбраны 4 характеристики: дифференциальная равномерность, нелинейность, минимальная алгебраическая степень и алгебраическая иммунность.

Также удалось построить подстановку на $\mathbb{F}_{2^6}$ с порядком дифференциальной равномерности 4:

$$(0, 2, 5, 4, 19, 41, 33, 42, 28, 31, 36, 58, 37, 32, 48, 9, 56, 50, 13, 20, 15, 60, 1, 12, 55, 11, 7, 35, 8, 47, 25, 39,$$ $$59, 18, 63, 10, 46, 24, 38, 17, 30, 61, 40, 29, 3, 45, 43, 22, 44, 62, 53, 34, 57, 21, 14, 54, 27, 26, 23, 49, 6, 16,$$ $$52, 51)$$
\begin{table}[h!]
\centering
\caption{Таблица 1}
\label{tab:permutation_comparision}
\begin{tabular}{|l|l|l|l|l|}
\hline
                                            & \textbf{$\pi_{AES}$} & \textbf{$\pi_{Kuznechik}$} & \textbf{$\pi_{Jimenez}$\cite{Jimenez}} & \textbf{$\pi$} \\ \hline
\textbf{Дифференциальная равномерность}     & 4            & 8                 & 6               & 6            \\ \hline
\textbf{Нелинейность}                       & 112          & 100               & 104             & 100          \\ \hline
\textbf{Минимальная алгебраическая степень} & 7            & 7                 & 7               & 7            \\ \hline
\textbf{Алгебраическая иммунность}          & 4            & 4                 & 4               & 4            \\ \hline
\end{tabular}
\end{table}

Исходный код вычислительных экспериментов можно найти в открытом репозитории \begin{center}
    https://github.com/Raz1el/APN-research
\end{center}
\end{abstractK}
\renewcommand\refname{\center \textnormal{\normalsize{ЛИТЕРАТУРА}}}
%Список литературы
\begin{thebibliography}{}
    \bibitem{LOS} Лось, А. Б.  Криптографические методы защиты информации для изучающих компьютерную безопасность : учебник для вузов / А. Б. Лось, А. Ю. Нестеренко, М. И. Рожков. — 2-е изд., испр. — Москва : Издательство Юрайт, 2025.
    \bibitem{Belov} А. Р. Белов, “Характеризация биективных APN-отображений в терминах расстояния между подгруппами симметрической группы”, ПДМ, 2023, № 60, 5–12
    \bibitem{Jimenez} R. A. de la Cruz Jimenez, “Constructing 8-bit permutations, 8-bit involutions and 8-bit orthomorphisms with almost optimal cryptographic parameters”, Матем. вопр. криптогр., 12:3 (2021), 89–124
    \bibitem{Hou} Hou~X.-D.~(2006). Affinity of permutations of $F_{2^n}$. {\sl Discret. Appl. Math.}. V.~{\bf 154}, P.~313--325.

\end{thebibliography}
\bf{Кураторы исследования:} \rm{\\
 Чижов Иван Владимирович~--- к.ф.-м.н., зам. по науке
руководителя лаборатории криптографии;
 \\
 Коломеец Николай Александрович~--- к.ф.-м.н., н.с. международного научно-образовательного Математического центра НГУ.
}

\end{document}